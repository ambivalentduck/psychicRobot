%%%%%%%%%%%%%%%%%%%%%%%%%%%%%%%%%%%%%%%%%%%%%%%%%%%%%%%%%%%%%%%%%%%%%%%%%%%%%%%%%%%%%%%%%%%%%%%%%%%%%%%%%%%%%%%%%%%%%%%%%%%%%%%%%%%%%%%%%%%%%%%%%%%%%%%%%%%
% This is just an example/guide for you to refer to when submitting manuscripts to Frontiers, it is not mandatory to use frontiers.cls nor frontiers.tex  %
% This will only generate the Manuscript, the final article will be typeset by Frontier after acceptance.                                                 %
%                                                                                                                                                         %
% When submitting your files, remember to upload this *tex file, the pdf generated with it, the *bib file (if bibliography is not within the *tex) and all the figures.
%%%%%%%%%%%%%%%%%%%%%%%%%%%%%%%%%%%%%%%%%%%%%%%%%%%%%%%%%%%%%%%%%%%%%%%%%%%%%%%%%%%%%%%%%%%%%%%%%%%%%%%%%%%%%%%%%%%%%%%%%%%%%%%%%%%%%%%%%%%%%%%%%%%%%%%%%%%

%%% Version 3.0 Generated 2014/12/19 %%%
%%% You will need to have the following packages installed: datetime, fmtcount, etoolbox, fcprefix, which are normally inlcuded in WinEdt. %%%
%%% In http://www.ctan.org/ you can find the packages and how to install them, if necessary. %%%

\documentclass{frontiersSCNS} % for Science, Engineering and Humanities and Social Sciences articles
%\documentclass{frontiersHLTH} % for Health articles
%\documentclass{frontiersFPHY} % for Physics articles

%\setcitestyle{square}
\usepackage{url,lineno}
\linenumbers

% Leave a blank line between paragraphs instead of using \\


\def\keyFont{\fontsize{8}{11}\helveticabold }
\def\firstAuthorLast{Horowitz {et~al.}} %use et al only if is more than 1 author
\def\Authors{Justin Horowitz\,$^1$, Tejas Madhavan\,$^1$, Christine Massie\,$^1$ and James Patton\,$^{1,2,*}$}
% Affiliations should be keyed to the author's name with superscript numbers and be listed as follows: Laboratory, Institute, Department, Organization, City, State abbreviation (USA, Canada, Australia), and Country (without detailed address information such as city zip codes or street names).
% If one of the authors has a change of address, list the new address below the correspondence details using a superscript symbol and use the same symbol to indicate the author in the author list.
\def\Address{$^{1}$Neuro-Machine Interaction Laboratory, University of Illinois at Chicago, Bionengineering, Chicago, Illinois, United States of America \\
$^{2}$Robotics Lab, Rehabilitation Institute of Chicago, Sensory Motor Performance Program, Chicago, Illinois, United States of America }
% The Corresponding Author should be marked with an asterisk
% Provide the exact contact address (this time including street name and city zip code) and email of the corresponding author
\def\corrAuthor{James Patton}
\def\corrAddress{Neuro-Machine Interaction Laboratory, University of Illinois at Chicago, Bionengineering, 851 S. Morgan Street, Chicago, Illinois, 60607-7052, United States of America}
\def\corrEmail{pattonj@uic.edu}




\begin{document}
\onecolumn
\firstpage{1}

\title[Realtime Feedback of Intended Trajectory]{Reaching is Better When You Get What You Want: Realtime Feedback of Intended Reaching Trajectory Despite an Unstable Environment}
\author[\firstAuthorLast ]{\Authors}
\address{}
\correspondance{}
\extraAuth{}% If there are more than 1 corresponding author, comment this line and uncomment the next one.
%\extraAuth{corresponding Author2 \\ Laboratory X2, Institute X2, Department X2, Organization X2, Street X2, City X2 , State XX2 (only USA, Canada and Australia), Zip Code2, X2 Country X2, email2@uni2.edu}
\topic{}% If your article is part of a Research Topic, please indicate here which.

\maketitle

%%%%%%%%%%%%%%%%%%%%%%%%%%%%%%%%%%%%%%%%%%%%%%%%%%%%%%%%%%%%%%%%%%%%%%%%%%%%%%%%%%%%%%%%%%%%%%%%%%%%%%%%%%%%%%%%%%%%%%%%%%%%%%%%%%%%%%%%%%%%%%%%%%%%%%%%%%%%%%%%%%%%%%%%%%%%%%%%%%%%%%%%%%%%%%%%%%%%%%%%%%%%%%%%%%%%%%%%%%%%%%%%%%%%%%%
%%% The sections below are for reference only.
%%%
%%% For Original Research Articles, Clinical Trial Articles, and Technology Reports the section headings should be those appropriate for your field and the research itself. It is recommended to organize your manuscript in the
%%% following sections or their equivalents for your field:
%%% Abstract, Introduction, Material and Methods, Results, and Discussion.
%%% Please note that the Material and Methods section can be placed in any of the following ways: before Results, before Discussion or after Discussion.
%%%
%%%For information about Clinical Trial Registration, please go to http://www.frontiersin.org/about/AuthorGuidelines#ClinicalTrialRegistration
%%%
%%% For Clinical Case Studies the following sections are mandatory: Abstract, Introduction, Background, Discussion, and Concluding Remarks.
%%%
%%% For all other article types there are no mandatory sections.
%%%%%%%%%%%%%%%%%%%%%%%%%%%%%%%%%%%%%%%%%%%%%%%%%%%%%%%%%%%%%%%%%%%%%%%%%%%%%%%%%%%%%%%%%%%%%%%%%%%%%%%%%%%%%%%%%%%%%%%%%%%%%%%%%%%%%%%%%%%%%%%%%%%%%%%%%%%%%%%%%%%%%%%%%%%%%%%%%%%%%%%%%%%%%%%%%%%%%%%%%%%%%%%%%%%%%%%%%%%%%%%%%%%%%%%

\begin{abstract}

%%% Leave the Abstract empty if your article falls under any of the following categories: Editorial Book Review, Commentary, Field Grand Challenge, Opinion or specialty Grand Challenge.
\section{}
%As a primary goal, the abstract should render the general significance and conceptual advance of the work clearly accessible to a broad readership. References should not be cited in the abstract.
Unstable and uncertain environments are an everyday problem that affects piloting and input devices in human-machine interaction. Here we experimentally evaluate intent feedback (IF), which estimates and displays the human operator’s underlying intent in real-time. IF is a filter that combines a model of the arm with position and force data to determine the intended position. Subjects performed targeted reaching motions while seeing either their hand position or their intent estimate as a cursor while they experienced white noise forces rendered by a robotic handle. We found significantly better reaching performance during force exposure using the estimated intent. Additionally, IF reduced estimated arm stiffness to about half that without IF, indicating a more relaxed state of operation. While visual distortions typically degrade performance and require an adaptation period to overcome, this particular distortion immediately enhances performance. In the future, this method could provide novel insights into the nature of control. IF might also be applied in driving and piloting applications to best follow a person’s desire in unpredictable or turbulent conditions.



\tiny
 \keyFont{ \section{Keywords:} Text Text Text Text Text Text Text Text } %All article types: you may provide up to 8 keywords; at least 5 are mandatory.
\end{abstract}

\section{Introduction}

% For Original Research Articles, Clinical Trial Articles, and Technology Reports the introduction should be succinct, with no subheadings.
Humans often interact with machines in uncertain and complicated environments, such as crowds and traffic, where they must contend with turbulence, moving obstacles, distractions, and disturbances. Despite our capacity to learn and adapt, some environments evolve too quickly or with too much uncertainty for meaningful learning. Sensory delays, sensory fusion, delays in sensory processing, and motor delays further confound our reactions. The nervous system intelligently solves many problems by planning ahead \citep{belen1967control}, yet in the face of uncertainties we often cannot adequately prevent errors. There is the possibility, however, to exploit additional information from instruments -- particularly fast and accurate force sensors -- that can measure human machine interactions. Combining sensors with filtering techniques makes it possible to determine a person’s underlying intent -- the motion they would have done had they not been disturbed. This intent provides new ways to understand the nature of control and provide novel feedback.

Recent work in our laboratory has attempted to outline a methodology for obtaining estimates of intent \citep{horowitz2015determining}. This method assumes a model of the dynamics and control of the human arm. Following manipulations of the equations of motion, the method integrates to find a unique estimate of intent. The algorithm recovers the trajectory a person intended to take, even if they were forced away from it due to environmental disturbances. This analysis has enabled us to show how some subjects sometimes alter their intent following exposure to unexpected force pulses.

A new question that arises is whether seeing one’s own intent, rather than what actually happens, may be useful. The intent extraction method can be streamlined to allow for real-time estimations of intent that can be presented to the subject as a cursor. Estimated intention may outperform the movement accuracy in the presence of unexpected disturbances. If so, such a method holds great promise in any situation where humans and machines interact as it enables the machine to give the human operator what they want. This human-machine collaboration could outperform what a person can do alone.

Displaying anything other than what truthfully happens is a distortion and a deceit. Like many other visual distortion experiments \citep{miles1980long, pine1996learning}, intent feedback (IF) introduces a visuomotor discrepancy that may be confusing to the nervous system. Preplanning a specific route may not be necessary, and instead the system might continuously react to the environmental disturbances until it reaches the goal. If people try to achieve a goal while minimizing some measure of cost, it is possible to compute a set of rules for reacting to the environment \citep{todorov2002optimal}. No specific intended route is needed when using optimal feedback control. While this modeling strategy has been very successful at explaining data, it fundamentally assumes that corrective actions will be taken in response to relevant disturbances. Goals can also be reached at minimal \textit{expected} cost by constructing -- and possibly updating -- a specific intended route. If no particular trajectory is intended, the nervous system could be unable to recognize IF.

While performance is the best indicator of IF's worth, changes in arm stiffness can provide supporting evidence that subjects are actually getting what they want. Arm stiffness is known to increase during exposure to instability \citep{franklin2003adaptation} and uncertainty \citep{takahashi2001impedance}. We anticipated that these changes might also be modulated by the presence of IF. Reducing the effects of environmental instability and presenting the subject with a signal already known to them should relax their arm and make it more compliant. We hypothesized that any elevated arm stiffness from noisy disturbances would decrease while subjects received IF.

In this paper we describe our streamlined method for real-time IF and then we expose subjects to IF in an unstable and unpredictable environment. We then compare their ability to perform goal-directed reaching using visual feedback of their hand position against IF. We hypothesized that IF should lead to better performance in the presence of force-based disturbances. Accordingly, we hypothesized that during random disturbances, the intent trajectory should deviate less than the hand trajectory.

\begin{methods}
\section{Material \& Methods}

\subsection{Intent Extraction}
The well-known motion control structure of \cite{shadmehr1994adaptive} relates arm trajectory, $q$, to desired arm trajectory, $q_d$, and any external disturbance, $E$ using physical parameters of the arm. To show how this model can be algebraically inverted to instead describe desired arm trajectory as a function of arm trajectory and external disturbance, we write it as a torque balance:
\begin{equation}
\underbrace{\overbrace{M(q)\ddot{q}}^{\text{Inertia}}+\overbrace{G(q,\dot{q})}^{\text{Coriolis, Centripal}}}_{\text{Plant}}+E=
\underbrace{\overbrace{\tau_{ff}}^{\text{Feedforward}}+\overbrace{\tau_{fb}}^{\text{Feedback}}}_\text{Controller}
\end{equation}
Where $M$ is the mass matrix, $q$ is the joint angles, $\dot{q}$ is joint angular velocities, $\ddot{q}$ is joint angular accelerations, and $G$ contains both Coriolis and centripetal effects. Typical applications solve this torque balance for $\ddot{q}$ and use a numerical differential equation solver to predict arm trajectory in the context of a disturbance of interest, a feedback model, and feedforward torques determined by inverse dynamics. Rather than test hypotheses regarding the learning, production, or composition of this feedforward torque, we instead solve for it:
\begin{equation}
\tau_{ff}=M(q)\ddot{q}+G(q,\dot{q})+E-\tau_{fb}
\end{equation}
Then we note that feedforward torque can have a one-to-one correspondence with desired acceleration, $\ddot{q}_d$:
\begin{equation}
\hat{M}(q_d)\ddot{q}_d+\hat{G}(q_d,\dot{q}_d)+\hat{E}=\tau_{ff}
\end{equation}
Hats ( $\hat{}$ ) denote the nervous system's best estimate of a physical quantity. Combining these expressions, suppressing state dependencies, and solving for $\ddot{q}_d$:
\begin{equation}
\ddot{q}_d=\hat{M}^{-1} \{M\ddot{q}+G-\hat{G}+E-\hat{E}-\tau_{fb}\}
\end{equation}
In this form, a differential equation solver can determine $q_d$ as it evolves in time if a few assumptions are made and conditions are met. First, $\hat{E}$ must be modeled or assumed, so we choose $\hat{E}=0$. In the presence of a zero mean white noise force disturbance its mean should be zero, but it is unlikely to be exactly zero and may reflect an average of only the last few exposures \citep{scheidt2001learning}. Next, the matrix $\hat{M}(q_d)$ must be invertible, but we can ensure this through our choice of workspace. Finally, feedback torque requires a model of arm impedance, which is known to be task-dependent \citep{gomi1998task} and may vary over the course of a reach \citep{niu2010temporal}. With no prior knowledge of arm impedance for this task-disturbance combination, we presumed the feedback torque model of \cite{shadmehr1994adaptive} anticipating that it is sufficiently accurate or easy to learn (experiment 1). The experiment was repeated with a lower stiffness estimate (experiment 2) to explore any dependence on this assumption.

\subsection{Apparatus}
A planar manipulandum (described in \cite{patton2004robot}) was programmed to minimize any friction or mass. The MATLAB XPC-TARGET package \citep{MATLAB:2008} was used to render this force environment at 1000 Hz and data were collected at 1000 Hz.  Visual feedback of cursor position was performed at 60 Hz using OpenGL. Closed-loop data transmission time (position measurement to completed rendering to recognition of rendering by the position measurement system) was less than 8 milliseconds, ensuring a visual delay less than one 60 Hz frame. Numerical simulation was performed in real-time using the GNU Scientific Library's odeiv2 driver with Runge-Kutta-Fehlberg (4,5) stepping\citep{gough2009gnu}.

\subsection{Human Subjects}
The human data trajectories analyzed here are drawn from sixteen subjects who gave informed consent in accordance with Northwestern University Institutional Review Board, which specifically approved this study and follows the principles expressed in the Declaration of Helsinki. Fourteen male and two female right-handed subjects (ages 21 to 30) performed the reaches with their right arm and were not compensated. Subjects' arm segment lengths were directly measured \textit{in situ} while body mass was self-reported.

\subsection{Experimental Design}
We chose center-out reaches of the right arm to visually-present targets 15 centimeters from the center chosen at $60^\circ$ intervals. Target selection was carried out pseudorandomly such that each outer target was visited 16 times in each of five blocks of 96 reaches each. During blocks 2 through 4, subjects experienced filtered white noise forces drawn from a white noise generator at 1000 Hertz with flat power spectral density of 1 Newton, and then passed through a 4th order low-pass Butterworth filter with cutoff $10 \pi$ radians per second. During block 3, cursor position indicated deduced intent. During all other blocks, the cursor position indicated hand position (Figure 2A).

\subsection{Dynamic simulation of arm and intended trajectories}
Anatomical landmarks and values from \cite{dempster1955space} and \cite{winter2009biomechanics} were used to estimate relationships between body mass and limb mass, limb length and limb center of mass, and limb mass and length and moment of inertia. Viscosity parameters, $K_d$, were taken from Shadmehr and Mussa-Ivaldi\cite{shadmehr1994adaptive}. Stiffness parameters were either taken from \cite{shadmehr1994adaptive} ($K_{P1}$, experiment 1) or estimated ($K_{P2}$, experiment 2). Expressed in Newton-Meters per Radian:
\begin{equation}
K_{P1}=
\begin{bmatrix}
15 & 6 \\
6 & 16
\end{bmatrix}
\quad
K_{P2}=
\begin{bmatrix}
8 & 2 \\
2 & 5
\end{bmatrix}
\end{equation}
To estimate this reduced stiffness, a pilot subject was asked to intend to remain still on the center target, $q_d$, while co-contracting as little as possible. $K_{P2}$ was then calculated from one minute of white noise forces, $E$, and joint angle traces, $q$ as the least squares solution to the system:
\begin{equation}
K_{P2}\big(q_d(t)-q(t)\big)=M\big(q(t)\big)\ddot{q}(t)+G(q(t),\dot{q}(t))+E(t)+K_D\dot{q}(t)
\end{equation}
where $K_{P2}$ is a 2-by-2 matrix while the state difference and torque are 2-by-60000 matrices.

\subsection{Metrics and Statistical Analysis}
Trajectories and forces were rotated such that movement and force parallel to the line connecting the previous target (the reach origin) and the presented target were along a \textit{progress} axis while perpendicular movement and force were along an \textit{error} axis. Reach onset was detected as the moment the cursor's distance from the center of the previous target first exceeded 1 centimeter. Maximum perpendicular error for a trajectory was the largest error magnitude within 250 milliseconds of reach onset. A scalar stiffness, $k$, was calculated for the \textit{error} axis during this same 250 milliseconds time span by linear regression:
\begin{equation}
F_e=m\ddot{e}+b\dot{e}+ke+F_O
\end{equation}
Force ($F_e$) and state ($\ddot{e}$, $\dot{e}$, $e$) were known. Mass ($m$), viscosity ($b$), and stiffness offset ($F_0$) terms were calculated, but discarded. While joint stiffness is usually described as a matrix, instantaneous endpoint stiffness in only the error direction is a scalar. This metric isolates stiffness in the error direction and facilitates statistical comparison between treatments and blocks. The paired Wilcoxon signed rank test was used to detect differences in maximum perpendicular error and stiffness between blocks and treatments at the 5\% significance level using the MATLAB statistics toolbox package \cite{MATLAB:2014}.
\end{methods}


\section{Results}
As expected, the model was able to deduce an intended trajectory and all subjects were able to utilize this estimate of their intent to perform targeted reaching while experiencing turbulent forces (Figure 1). The following sections address the quality of this reaching.

\subsection*{Reaching Accuracy}
Even if subjects were capable of using IF to perform a reaching task, a poor estimate would have lead to poor performance. Indeed, the hand significantly outperformed our intent estimate when the force disturbance is not present in block 1 (Figure 2D). Model and sensor inaccuracies degrade performance and result in an error of approximately 5 millimetres. In the presence of a force disturbance in block 2, the hand loses its advantage as model and sensor inaccuracies are balanced by some degree of cancellation of the disturbance. In block 3 where visual feedback of the intent is given instead of hand, the intent significantly outperforms the hand. For this intent estimate to be useful, subjects should perform better when using feedback of the intent than when using feedback of the hand. Comparing the hand's performance in block 2 to the intent's performance in block 3, subjects performed significantly better using their estimated intent. Finally, no significant differences were present across the two experiments. Performance was relatively insensitive to this change in the stiffness model used to deduce the intent.

\subsection*{Endpoint Stiffness}
While accuracy is desirable, it can be attained by co-contracting muscles to increase arm stiffness at the cost of metabolic effort \citep{gribble2003role, takahashi2001impedance}. Subjects stiffened significantly in response to white noise forces as seen by comparing stiffness in blocks 1 and 2 (Figure 3C). Subjects' stiffness significantly decreased when using IF replaced the hand position as their cursor, as seen by comparing blocks 2 and 3. This decrease does not return stiffness to undisturbed levels, and it remains even after feedback of the hand resumes as seen by comparing blocks 3 and 4. Stiffness may have decreased with duration of exposure to uncertain disturbances \citep{takahashi2001impedance}. Subjects did not appear to have adjusted their stiffness to match the stiffness used in the estimation as subject and model stiffnesses were significantly different in experiment 2.

\section{Discussion}

Although exposure to random forces hindered subjects’ reaching accuracy and increased their arm stiffness, replacing the veridical feedback with IF improved accuracy and decreased stiffness. While other visual distortions typically degrade performance and require an adaptation period to overcome, IF immediately enhanced performance. The performance variability inherent in white noise force disturbances complicates observation of learning, but there were no obvious differences in performance between the first and final exposures to IF.The simplest explanation for this is that IF approximates a signal already known to the brain: the path planned for the hand. In addition to its promise in performance enhancement, IF represents a novel means of revealing and studying the mechanisms of motor planning and motor control.

While IF alleviated the increased stiffness caused by exposure to random forces, stiffness remained significantly above baseline levels. Many explanations are reasonable. In particular, we might have hypothesized that subjects would adapt their own arm stiffness to decrease conflict with the stiffness model used to estimate their intent and thereby increase the accuracy of the estimate, but the data did not support this conclusion. Alternatively, inaccuracy and incompleteness of the models used might have resulted in an estimate of intent less accurate than subjects might desired. Since noise and performance inaccuracy can both be addressed by co-contraction, this may account for the residual stiffness. Finally, as subjects were not cued regarding the onset or removal of IF, the residual co-contraction may have been a precaution against the resumption of veridical feedback.

The ease with which subjects could make use of their estimated intent provides strong preliminary evidence that a specific intended trajectory is computed for the hand even when reaching in a highly variable environment. While recent work has identified kinematic constraints unnecessary for a task \citep{mistry2013optimal}, this is the first direct evidence that the entire trajectory is controlled even in the absence of specific instructions or constraints. It remains unclear when the intended trajectory is computed, but variability in the desired trajectory in phases 2 and 4 suggests that the trajectory is planned or at least can be replanned mid-movement. Certainly, it is not finalized before the onset of movement and is not a "strict invariant."

IF can facilitate human-machine collaboration and artificial performance augmentation by enabling the machine to preserve an operator's intent while cancelling unexpected disturbances from the environment. This would reduce the demand on the human operator and increase performance -- especially in environments with rapidly changing conditions. This assistance goes beyond environment cancellation by also accounting for any errors the operator might make based on their expectation of disturbance.

While IF holds tremendous promise, it also has strong limitations. IF is entirely dependent on the accuracy of the models used. While we were able to leverage measurements of cadaver anthropometrics, average tendencies do not capture individual variability. Similarly, the stiffness model of \cite{shadmehr1994adaptive} appears to have accurately captured the mean tendency (figure 3, bottom left) without accounting for variation among individuals or variation over time. Techniques that could estimate stiffness trajectories, perhaps even in real-time, would greatly increase the accuracy and utility of IF.

IF benefited subjects moving in a changing, uncertain environment. In addition to increasing subjects' accuracy, IF may have allowed subjects to reach their goals with less effort as arm stiffness decreased. While IF is limited by the accuracy of the models used, many candidate models are available and may outperform the simple model we investigated here. IF provides a novel form of feedback that may facilitate new insights into the nature of motor control and allows a machine to collaborate more effectively with a human user.


\section*{Disclosure/Conflict-of-Interest Statement}
%Frontiers follows the recommendations by the International Committee of Medical Journal Editors (http://www.icmje.org/ethical_4conflicts.html) which require that all financial, commercial or other relationships that might be perceived by the academic community as representing a potential conflict of interest must be disclosed. If no such relationship exists, authors will be asked to declare that the research was conducted in the absence of any commercial or financial relationships that could be construed as a potential conflict of interest. When disclosing the potential conflict of interest, the authors need to address the following points:
%•	Did you or your institution at any time receive payment or services from a third party for any aspect of the submitted work?
%•	Please declare financial relationships with entities that could be perceived to influence, or that give the appearance of potentially influencing, what you wrote in the submitted work.
%•	Please declare patents and copyrights, whether pending, issued, licensed and/or receiving royalties relevant to the work.
%•	Please state other relationships or activities that readers could perceive to have influenced, or that give the appearance of potentially influencing, what you wrote in the submitted work.

The authors declare that the research was conducted in the absence of any commercial or financial relationships that could be construed as a potential conflict of interest.

\section*{Author Contributions}
%When determining authorship the following criteria should be observed:
%•	Substantial contributions to the conception or design of the work; or the acquisition, analysis, or interpretation of data for the work; AND
%•	Drafting the work or revising it critically for important intellectual content; AND
%•	Final approval of the version to be published ; AND
%•	Agreement to be accountable for all aspects of the work in ensuring that questions related to the accuracy or integrity of any part of the work are appropriately investigated and resolved.
%Contributors who meet fewer than all 4 of the above criteria for authorship should not be listed as authors, but they should be acknowledged. (http://www.icmje.org/roles_a.html)

All authors designed the work, analyzed the data, drafted and revised the manuscript, approved the final version, and agree to be held accountable for all aspects. TM and CM collected the data.

\section*{Acknowledgments}
The authors wish to thank Yazan Abdel Majeed for his review of the preliminary manuscript. We also thank the community  of the Robotics Lab at the Rehabilitation Institute of Chicago for critical commentary during the genesis of this paper and the work leading up to it.


\paragraph{Funding\textcolon} Funded By NIH R01-NS053606 and NIDRR H133E120010.


\bibliographystyle{frontiersinSCNS} % for Science, Engineering and Humanities and Social Sciences articles, for Humanities and Social Sciences articles please include page numbers in the in-text citations
%\bibliographystyle{frontiersinHLTH&FPHY} % for Health and Physics articles
\bibliography{paper2.bib}

%%% Upload the *bib file along with the *tex file and PDF on submission if the bibliography is not in the main *tex file

\section*{Figures}

%%% Use this if adding the figures directly in the mansucript, if so, please remember to also upload the files when submitting your article
%%% There is no need for adding the file termination, as long as you indicate where the file is saved. In the examples below the files (logo1.jpg and logo2.eps) are in the Frontiers LaTeX folder
%%% If using *.tif files convert them to .jpg or .png

\begin{figure}[h!]
\begin{center}
\includegraphics[width=10cm]{Yplots.eps}
\end{center}
\textbf{\refstepcounter{figure}\label{fig:01} Figure \arabic{figure}.}{ Typical subjects made center-out targeted reaching motions under experimentally varied force and feedback conditions. Subjects used feedback of either hand motions (black lines) or estimated intent (red lines) to complete these reaches. Intent was estimated using either the stiffness model of \cite{shadmehr1994adaptive} (Experiment 1) or a reduced stiffness (Experiment 2) to explore any dependence of reaching stiffness or accuracy on this assumption. The white noise force disturbance was designed to be unpredictable in order to minimize any effect of learning. }
\end{figure}

\begin{figure}[h!]
\begin{center}
\includegraphics[width=10cm]{error.eps}
\end{center}
\textbf{\refstepcounter{figure}\label{fig:02} Figure \arabic{figure}.}{ Subjects' reaching accuracy depended on the presence of force disturbance and the contents of visual feedback (A). Maximum deviation from straight-line reaching calculated during the first 250 milliseconds after the onset of movement revealed that turbulent force disturbance degraded reaching performance in both hand (B) and intent (C). Comparisons within treatment conditions revealed a significant advantage for the hand when force disturbance is not present and a significant advantage for intent when feedback of intent was used during exposure to turbulent forces. Performance using intent during turbulence was significantly better than performance using the hand (D). Significance among comparisons is denoted by an asterisk. Error did not signficantly depend on the choice between stiffness model of \cite{shadmehr1994adaptive} (experiment 1, blue) or a reduced stiffness (experiment 2, red) to determine intent.}
\end{figure}


\begin{figure}[h!]
\begin{center}
\includegraphics[width=10cm]{stiffness.eps}
\end{center}
\textbf{\refstepcounter{figure}\label{fig:03} Figure \arabic{figure}.}{ Subjects' arm stiffness depended on the presence of force disturbance and the contents of visual feedback (A). Arm stiffness calculated by linear regression during the first 250 milliseconds after the onset of movement revealed that turbulent force disturbance increased arm stiffness (B). Comparisons between treatment conditions revealed that exposure to turbulent forces caused by arm to stiffen significantly, but feedback of intent could significantly reduce this increase (C). Stiffness did not signficantly depend on the choice between the stiffness model of \cite{shadmehr1994adaptive} (experiment 1, blue) or a reduced stiffness (experiment 2, red) to determine intent.}
\end{figure}

%%% If you don't add the figures in the LaTeX files, please upload them when submitting the article.

%%% Frontiers will add the figures at the end of the provisional pdf automatically %%%

%%% The use of LaTeX coding to draw Diagrams/Figures/Structures should be avoided. They should be external callouts including graphics.

\end{document}
