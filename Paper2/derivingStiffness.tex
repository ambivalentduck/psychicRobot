\documentclass[10pt]{article}

% amsmath package, useful for mathematical formulas
\usepackage{amsmath}
% amssymb package, useful for mathematical symbols
\usepackage{amssymb}

% graphicx package, useful for including eps and pdf graphics
% include graphics with the command \includegraphics
\usepackage{graphicx}
\usepackage{color} 

% Text layout
\topmargin 0.0cm
\oddsidemargin 0.5cm
\evensidemargin 0.5cm
\textwidth 16cm 
\textheight 21cm

% Bold the 'Figure #' in the caption and separate it with a period
% Captions will be left justified
\usepackage[labelfont=bf,labelsep=period,justification=raggedright]{caption}

\pagestyle{myheadings}

\begin{document}

% Title must be 150 characters or less
Define terms: $x$ is hand, $y$ is intent. Capital $F$, $X$ and $Y$ are Laplace transforms. $M$, $B$, and $K$ are mass, damping, and stiffness. $f$ is some externally applied force. Hats ( $\hat{ }$ ) are estimates from an internal model.
\begin{equation}
M\ddot{x}+B\dot{x}+Kx+f=\hat{M}\ddot{y}+\hat{B}\dot{y}+\hat{K}y
\end{equation}
Drop the hats because mass and damping are more or less correctly estimated and we're going to re-represent stiffness misestimation later. Also, switch to the Laplace domain:
\begin{equation}
(Ms^2+Bs+K)X+F=(Ms^2+Bs+K)Y
\end{equation}  
\begin{equation}
F=(Ms^2+Bs+K)(Y-X)
\end{equation}  
Define the quantity, $E=Y-X$, which is essentially error:
\begin{equation}
F=(Ms^2+Bs+K)E
\end{equation} 
Since X and F are presumably matters of record, Y and the arm parameters are the only unknowns, so we can extract. Now, extract with some stiffness estimation error, $\epsilon$. This means that the stiffness used in the extraction, $K_e=K+\epsilon$.
\begin{equation}
F=(Ms^2+Bs+K+\epsilon)E_e
\end{equation}
So the extracted error, $E_e$ is \textit{not} the same as $E$ because of $\epsilon$. But they're related through $F$:
\begin{equation}
(Ms^2+Bs+K)E=(Ms^2+Bs+K+\epsilon)E_e
\end{equation} 
\begin{equation}
(Ms^2+Bs+K)(E-E_e)=\epsilon E_e
\end{equation} 
\begin{equation}
\frac{E-E_e}{E_e}=\frac{\epsilon}{Ms^2+Bs+K}
\end{equation} 
Generally speaking, this next equation is probably a fair place to stop:
\begin{equation}
\frac{E}{E_e}=1+\frac{\epsilon}{Ms^2+Bs+K}
\end{equation}
But we can re-write it a bit:
\begin{equation}
\frac{Y-X}{Y_e-X}=1+\frac{\epsilon}{Ms^2+Bs+K}
\end{equation} 
\begin{equation}
Y-Y_e=\frac{\epsilon}{Ms^2+Bs+K}(Y_e-X)
\end{equation}
\begin{equation}
\frac{Y-Y_e}{Y_e-X}=\frac{\epsilon}{Ms^2+Bs+K}
\end{equation}
Stiffness misestimation creates a transient in the extraction whose frequency response is entirely defined by the true dynamics of the system. Because we know $E_e$, the sum of $\epsilon$ and $K$, we have two unknowns: $Y$ and $\epsilon$. A difference of two different extractions using equation 11 does not contain this term:
\begin{equation}
(Y-Y_{1e})-(Y-Y_{2e})=(\frac{\epsilon_1}{Ms^2+Bs+K}(Y_{1e}-X))-(\frac{\epsilon_2}{Ms^2+Bs+K}(Y_{2e}-X))
\end{equation}
\begin{equation}
Y_{2e}-Y_{1e}=\frac{\epsilon_1 Y_{1e}-\epsilon_2 Y_{2e}+(\epsilon_2-\epsilon_1)X}{Ms^2+Bs+K}=\epsilon_1 \frac{Y_{1e}-X}{Ms^2+Bs+K}- \epsilon_2 \frac{Y_{2e}-X}{Ms^2+Bs+K}
\end{equation}
Because we know the difference in $K_e$ values used for the two extractions ($K_{1e}$ and $K_{2e}$), we know the difference in $\epsilon$ values already.
\begin{equation}
Y_{2e}-Y_{1e}=\frac{\epsilon_1 Y_{1e}-\epsilon_2 Y_{2e}+(\epsilon_2-\epsilon_1)X}{Ms^2+Bs+K}=\epsilon_1 \frac{Y_{1e}-X}{Ms^2+Bs+K}- (\epsilon_1+K_{2e}-K_{1e}) \frac{Y_{2e}-X}{Ms^2+Bs+K}
\end{equation}
Because $Y_{1e}$, $Y_{2e}$, and $X$ are known, we can solve for our unknown. In order to deal with the dynamics in the denominators, a slightly lagged value of $K$ should be fine (1 samples = 1 ms, $K_{next}=K_{now}-\epsilon_1$). This implies that dynamic tracking of stiffness should be possible in real time. Convergence properties remain unexamined.
\end{document}