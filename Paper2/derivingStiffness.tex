\documentclass[10pt]{article}

% amsmath package, useful for mathematical formulas
\usepackage{amsmath}
% amssymb package, useful for mathematical symbols
\usepackage{amssymb}

% graphicx package, useful for including eps and pdf graphics
% include graphics with the command \includegraphics
\usepackage{graphicx}
\usepackage{color} 

% Text layout
\topmargin 0.0cm
\oddsidemargin 0.5cm
\evensidemargin 0.5cm
\textwidth 16cm 
\textheight 21cm

% Bold the 'Figure #' in the caption and separate it with a period
% Captions will be left justified
\usepackage[labelfont=bf,labelsep=period,justification=raggedright]{caption}

\pagestyle{myheadings}

\begin{document}

% Title must be 150 characters or less
Define terms: $x$ is hand, $y$ is intent. Capital $F$, $X$ and $Y$ are Laplace transforms. $M$, $B$, and $K$ are mass, damping, and stiffness. $f$ is some externally applied force. Hats ( $\hat{ }$ ) are estimates from an internal model.
\begin{equation}
M\ddot{x}+B\dot{x}+Kx+f=\hat{M}\ddot{y}+\hat{B}\dot{y}+\hat{K}y
\end{equation}
Drop the hats because mass and damping are more or less correctly estimated and we're going to re-represent stiffness misestimation later. Also, switch to the Laplace domain:
\begin{equation}
(Ms^2+Bs+K)X+F=(Ms^2+Bs+K)Y
\end{equation}  
\begin{equation}
F=(Ms^2+Bs+K)(Y-X)
\end{equation}  
Define the quantity, $E=Y-X$, which is essentially error:
\begin{equation}
F=(Ms^2+Bs+K)E
\end{equation} 
Since $M$, $B$, $X$ and $F$ are presumably matters of record, $Y$ and $K$ are the only unknowns. We can extract, but we do so with some as yet unknown stiffness estimation error, $\epsilon$. This means that the stiffness used in the extraction, $K_e=K+\epsilon$.
\begin{equation}
F=(Ms^2+Bs+K+\epsilon)E_e
\end{equation}
So the extracted error, $E_e$ is \textit{not} the same as $E$ because of $\epsilon$. But they're related through $F$:
\begin{equation}
(Ms^2+Bs+K)E=(Ms^2+Bs+K+\epsilon)E_e
\end{equation} 
\begin{equation}
(Ms^2+Bs+K)(E-E_e)=\epsilon E_e
\end{equation} 
\begin{equation}
\frac{E-E_e}{E_e}=\frac{\epsilon}{Ms^2+Bs+K}
\end{equation} 
Generally speaking, this next equation is probably a fair place to stop:
\begin{equation}
\frac{E}{E_e}=1+\frac{\epsilon}{Ms^2+Bs+K}
\end{equation}
But we can re-write it a bit:
\begin{equation}
\frac{Y-X}{Y_e-X}=1+\frac{\epsilon}{Ms^2+Bs+K}
\end{equation} 
\begin{equation}
Y-Y_e=\frac{\epsilon}{Ms^2+Bs+K}(Y_e-X)
\end{equation}
\begin{equation}
\frac{Y-Y_e}{Y_e-X}=\frac{\epsilon}{Ms^2+Bs+K}
\end{equation}
Stiffness misestimation creates a transient in the extraction whose frequency response is entirely defined by the true dynamics of the system. Because we know the value of $K$ used for each extraction, $K_e=\epsilon+K$, we have two unknowns: $Y$ and $\epsilon$. A difference of two different extractions using equation 11 does not contain the $Y$ term:
\begin{equation}
(Y-Y_{1e})-(Y-Y_{2e})=\epsilon_1 \frac{Y_{1e}-X}{Ms^2+Bs+K}- \epsilon_2 \frac{Y_{2e}-X}{Ms^2+Bs+K}
\end{equation}
Because we know the difference in $K_e$ values used for the two extractions ($K_{1e}$ and $K_{2e}$), we know the difference in $\epsilon$ values already.
\begin{equation}
Y_{2e}-Y_{1e}=\epsilon_1 \frac{Y_{1e}-X}{Ms^2+Bs+K}-(\epsilon_1+K_{2e}-K_{1e}) \frac{Y_{2e}-X}{Ms^2+Bs+K}
\end{equation}
\begin{equation}
Y_{2e}-Y_{1e}=\epsilon_1 \frac{Y_{1e}-Y_{2e}}{Ms^2+Bs+K}-(K_{2e}-K_{1e}) \frac{Y_{2e}-X}{Ms^2+Bs+K}
\end{equation}
Because $Y_{1e}$, $Y_{2e}$, and $X$ are known, we can solve for our unknown. In order to deal with the dynamics in the denominators, a slightly lagged value of $K$ should be fine (1 sample = 1 ms delay, $K_{next}=K_{now}-\epsilon_1$). This implies that dynamic tracking of stiffness should be possible in real time. Convergence properties remain unexamined but this might just start and stay at your initial guess (and probably will).
\begin{equation}
(Ms^2+Bs+K)(Y_{2e}-Y_{1e})=\epsilon_1 (Y_{1e}-Y_{2e})-(K_{2e}-K_{1e})(Y_{2e}-X)
\end{equation}
\begin{equation}
(Ms^2+Bs)(Y_{2e}-Y_{1e})=(K+\epsilon_1)(Y_{1e}-Y_{2e})-(K_{2e}-K_{1e})(Y_{2e}-X)
\end{equation}
\begin{equation}
(Ms^2+Bs+K+\epsilon_1)(Y_{2e}-Y_{1e})=(K_{2e}-K_{1e})(X-Y_{2e})
\end{equation}
\begin{equation}
\frac{Y_{1e}-Y_{2e}}{Y_{2e}-X}=\frac{K_{2e}-K_{1e}}{Ms^2+Bs+K_{1e}}
\end{equation}
\begin{equation}
Y_{1e}-Y_{2e}=\frac{K_{2e}-K_{1e}}{Ms^2+Bs+K_{1e}}(Y_{2e}-X)
\end{equation}
Back to zero unknowns, which basically dooms equation 15. Notice that this is equivalent to eq 12.
\begin{equation}
\frac{Y-Y_{2e}}{Y_{2e}-X}-\frac{Y_{1e}-Y_{2e}}{Y_{2e}-X}=\frac{\epsilon_2}{Ms^2+Bs+K}-\frac{K_{2e}-K_{1e}}{Ms^2+Bs+K_{1e}}
\end{equation}
\begin{equation}
\frac{Y-Y_{1e}}{Y_{2e}-X}=\frac{\epsilon_2}{Ms^2+Bs+K}-\frac{K_{2e}-K_{1e}}{Ms^2+Bs+K_{1e}}
\end{equation}

\newpage
What if you know that $Y=720/s^6-360/s^5+60/s^4$ because of minimum jerk submovements?
\begin{equation}
Y_e=720/s^6-360/s^5+60/s^4-\frac{K_e-K}{Ms^2+Bs+K_e}(Y_e-X)
\end{equation}
\begin{equation}
(Ms^2+Bs+K_e)Y_e=720(M/s^4+B/s^5+K_e/s^6)-360(M/s^3+B/s^4+K/s^5)+60(M/s^2+B/s^3+K/s^4)+(K-K_e)(Y_e-X)
\end{equation}
\begin{equation}
(Ms^2+Bs+K_e)Y_e=720K_e/s^6+(720K-(M/s^4+B/s^5+K_e/s^6)-360(M/s^3+B/s^4+K/s^5)+60(M/s^2+B/s^3+K/s^4)+(K-K_e)(Y_e-X)
\end{equation}


\newpage
Show really quickly why you can't solve eq 16 for K:
\begin{equation}
(Ms^2+Bs+K)(Y_{2e}-Y_{1e})=(K_{1e}-K) (Y_{1e}-Y_{2e})-(K_{2e}-K_{1e})(Y_{2e}-X)
\end{equation}
\begin{equation}
(Ms^2+Bs)(Y_{2e}-Y_{1e})+K(Y_{2e}-Y_{1e})+K(Y_{1e}-Y_{2e})=K_{1e}(Y_{1e}-Y_{2e})-(K_{2e}-K_{1e})(Y_{2e}-X)
\end{equation}
\begin{equation}
(Ms^2+Bs)(Y_{2e}-Y_{1e})=K_{1e}(Y_{1e}-Y_{2e})-(K_{2e}-K_{1e})(Y_{2e}-X)
\end{equation}
As seen in eq 19, K actually vanishes. Because extractions are linearly dependent on one another, sums and differences contain no information you don't already have. Differences wipe out your two unknowns while sums contain them. There's an interest sum involving equal and opposite $\epsilon$ though. Snagging eq 11:
\begin{equation}
Y-Y_e=\frac{\epsilon}{Ms^2+Bs+K}(Y_e-X)
\end{equation}
\begin{equation}
Y_e=Y-\frac{\epsilon}{Ms^2+Bs+K}(Y_e-X)
\end{equation}
\begin{equation}
Y_1+Y_2=2Y-\frac{\epsilon_1}{Ms^2+Bs+K}(Y_1-X)-\frac{\epsilon_2}{Ms^2+Bs+K}(Y_2-X)
\end{equation}
Choose $\epsilon_1=-\epsilon_2$:
\begin{equation}
Y_1+Y_2=2Y+\frac{\epsilon_1}{Ms^2+Bs+K}(Y_2-X)-\frac{\epsilon_1}{Ms^2+Bs+K}(Y_1-X)
\end{equation}
\begin{equation}
Y_1+Y_2=2Y+\epsilon_1\frac{(Y_2-X)-(Y_1-X)}{Ms^2+Bs+K}
\end{equation}
\begin{equation}
Y_1+Y_2=2Y+\epsilon_1\frac{Y_2-Y_1}{Ms^2+Bs+K}
\end{equation}
\begin{equation}
Y_1+Y_2=2Y+\epsilon_1((K_{1e}-K) (Y_{1e}-Y_{2e})-(K_{2e}-K_{1e})(Y_{2e}-X))
\end{equation}




\newpage
Jumping off from the difference of two extractions containing zero unknowns, let's add a new unknown, $\alpha$. So the setup we want is $Y_{1e}-\alpha(Y_{2e}-Y_{1e})=Y+...$ where that ellipsis is a pile of knowns and $\alpha$. Because this is a fudge factor and not a physical parameter, it might allow us to not make weird assumptions about $Y$.
\begin{equation}
Y_{1e}-Y_{2e}=\frac{K_{2e}-K_{1e}}{Ms^2+Bs+K_{1e}}(Y_{2e}-X)
\end{equation}
\begin{equation}
Y_{1e}=Y-\frac{K_{1e}-K}{Ms^2+Bs+K_{1e}}(Y_{1e}-X)
\end{equation}
\begin{equation}
Y_{1e}-\alpha(Y_{1e}-Y_{2e})=Y-\frac{K_{1e}-K}{Ms^2+Bs+K_{1e}}(Y_{1e}-X)-\alpha\frac{K_{2e}-K_{1e}}{Ms^2+Bs+K_{1e}}(Y_{2e}-X)
\end{equation}
\begin{equation}
(Ms^2+Bs+K_{1e})(Y_{1e}-\alpha(Y_{1e}-Y_{2e})-Y)=-(K_{1e}-K)(Y_{1e}-X)-\alpha(K_{2e}-K_{1e})(Y_{2e}-X)
\end{equation}
Going off on a tangent to entertain $\alpha=1$, what happens?
\begin{equation}
(Ms^2+Bs+K_{1e})(Y_{2e}-Y)=-(K_{1e}-K)(Y_{1e}-X)-(K_{2e}-K_{1e})(Y_{2e}-X)
\end{equation}
It looks like we've achieved a situation where $Y$ and $K$ are not multiplying one another.
\begin{equation}
(Ms^2+Bs+K_{1e})Y_{2e}+(K_{2e}-K_{1e})(Y_{2e}-X)+K_{1e}(Y_{1e}-X)=(Y_{1e}-X)K+(Ms^2+Bs+K_{1e})Y
\end{equation}
So this is exactly what we've been looking for: knowns and constants on the left, $K$ and $Y$ on the right and multiplied only by knowns.

I messed up the math, try this route again later.



\newpage
\begin{equation}
Y=Y_e+\frac{\epsilon}{Ms^2-Bs+K_e-\epsilon}(Y_e-X)
\end{equation} 
Three equations and three unknowns $Y(t=-1)$, $Y(t=1)$, $\epsilon$
\begin{subequations}
\begin{align}
Y(t=-1)&=Y_e(t=-1)+\frac{\epsilon}{Ms^2-Bs+K_e-\epsilon}(Y_e(t=-1)-X(t=-1)) \\
\frac{Y(t=1)+Y(t=-1)}{2}&=Y_e(t=0)+\frac{\epsilon}{Ms^2-Bs+K_e-\epsilon}(Y_e(t=0)-X(t=0)) \\
Y(t=1)&=Y_e(t=1)+\frac{\epsilon}{Ms^2-Bs+K_e-\epsilon}(Y_e(t=1)-X(t=1))
\end{align}
\end{subequations}

That's three equations in three unknowns, but what if you go with two instead?
\begin{equation}
Y=Y_1+\frac{\epsilon}{Ms^2-Bs+K_e-\epsilon}(Y_e-X)
\end{equation} 


\newpage

\begin{equation}
F=(Ms^2+Bs+K_e-\epsilon)(Y-X)
\end{equation}
\begin{equation}
F=(Ms^2+Bs+K_e)(Y_e-X)
\end{equation}

\newpage
\textbf{Tactic: why does some weighted difference of three extractions work?}

Quickly define $D=Ms^2+Bs+K$ and take the difference of three extractions.
\begin{equation}
(Y-X)-((Y-X)-(Y-X))=Y_1-X+\frac{\epsilon_1}{D}-((Y_2-X+\frac{\epsilon_2}{D})-(Y_3-X+\frac{\epsilon_3}{D}))
\end{equation} 
\begin{equation}
Y=Y_1-Y_2+Y_3+\frac{\epsilon_1(Y_1-X)-\epsilon_2(Y_2-X)+\epsilon_3(Y_3-X)}{D}
\end{equation} 
\begin{equation}
Y=Y_1-Y_2+Y_3+\frac{\epsilon_1(Y_1-X)-(K_2-K_1+\epsilon_1)(Y_2-X)+(K_3-K_1+\epsilon_1)(Y_3-X)}{D}
\end{equation} 
\begin{equation}
(Ms^2+Bs+K_1-\epsilon_1)Y=(Ms^2+Bs+K_1-\epsilon_1)(Y_1-Y_2+Y_3)+\epsilon_1(Y_1-X)-(K_2-K_1+\epsilon_1)(Y_2-X)+(K_3-K_1+\epsilon_1)(Y_3-X)
\end{equation} 




\begin{equation}
Y=\frac{K_{2e}-K_{1e}}{Ms^2+Bs+K_{1e}}(Y_{2e}-X)+Y_3+\frac{\epsilon_1(Y_1-X)-(K_2-K_1+\epsilon_1)(Y_2-X)+(K_3-K_1+\epsilon_1)(Y_3-X)}{D}
\end{equation} 




\begin{equation}
Y-Y_e=\frac{\epsilon}{Ms^2+Bs+K_e-\epsilon}(Y_e-X)
\end{equation}
\begin{equation}
(Ms^2+Bs+K_e-\epsilon)(Y-Y_e)=\epsilon(Y_e-X)
\end{equation}
\begin{equation}
(Ms^2+Bs+K_e)Y_e-(Ms^2+Bs+K_e-\epsilon)Y=\epsilon X
\end{equation}
\begin{equation}
(Ms^2+Bs+K_e)(Y_e-Y)=\epsilon(X+Y)
\end{equation}


\newpage
\textbf{It wasn't this:}

Maybe it's three equations in three unknowns ($\epsilon_1$, $\epsilon_2$, and $Y$)?
\begin{subequations}
\begin{align}
K_{2e}-K_{1e} &= \epsilon_2 - \epsilon_1 \\
Y_{1e}&=Y-\frac{\epsilon_1}{Ms^2+Bs+K_{1e}-\epsilon_1}(Y_{1e}-X) \\
Y_{2e}&=Y-\frac{\epsilon_2}{Ms^2+Bs+K_{2e}-\epsilon_2}(Y_{2e}-X)
\end{align}
\end{subequations}
If so, can we somehow rearrange our way out of this?
\begin{subequations}
\begin{align}
K_{2e}-K_{1e} &= \epsilon_2 - \epsilon_1 \\
(Ms^2+Bs+K_{1e})Y_{1e}&=(Ms^2+Bs+K_{1e}-\epsilon_1)Y-\epsilon_1 X \\
(Ms^2+Bs+K_{2e})Y_{2e}&=(Ms^2+Bs+K_{2e}-\epsilon_2)Y-\epsilon_2 X
\end{align}
\end{subequations}
\begin{subequations}
\begin{align}
K_{2e}-K_{1e} &= \epsilon_2 - \epsilon_1 \\
(Ms^2+Bs+K_{1e})Y_{1e}&=(Ms^2+Bs+K_{1e}-\epsilon_1)Y-\epsilon_1 X \\
(Ms^2+Bs+K_{2e})Y_{2e}&=(Ms^2+Bs+K_{1e}-\epsilon_1)Y-\epsilon_2 X
\end{align}
\end{subequations}
Down to two equations in two unknowns:
\begin{subequations}
\begin{align}
(Ms^2+Bs+K_{1e})Y_{1e}&=(Ms^2+Bs+K_{1e}-\epsilon_1)Y-\epsilon_1 X \\
(Ms^2+Bs+K_{2e})Y_{2e}&=(Ms^2+Bs+K_{1e}-\epsilon_1)Y-(K_{2e}-K_{1e}+\epsilon_1) X
\end{align}
\end{subequations}
\begin{equation}
\frac{(Ms^2+Bs+K_{1e})Y_{1e}+\epsilon_1 X}{(Ms^2+Bs+K_{1e}-\epsilon_1)}=Y
\end{equation}
\begin{equation}
(Ms^2+Bs+K_{2e})Y_{2e}=(Ms^2+Bs+K_{1e}-\epsilon_1)(\frac{(Ms^2+Bs+K_{1e})Y_{1e}+\epsilon_1 X}{(Ms^2+Bs+K_{1e}-\epsilon_1)})-(K_{2e}-K_{1e}+\epsilon_1) X
\end{equation}
\begin{equation}
(Ms^2+Bs+K_{2e})Y_{2e}=(Ms^2+Bs+K_{1e})Y_{1e}-(K_{2e}-K_{1e}) X
\end{equation}


\end{document}